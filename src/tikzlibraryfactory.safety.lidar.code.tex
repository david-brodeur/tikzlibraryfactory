% tikzlibraryfactory.safety.lidar.code.tex
\makeatletter
% If \ProvidesTikzLibrary isn't defined yet, define it as an alias for \ProvidesPackage
\@ifundefined{ProvidesTikzLibrary}{%
  \def\ProvidesTikzLibrary#1[#2]{%
    \ProvidesFile{tikzlibrary#1}[#2]%
  }%
}{}
\makeatother

\ProvidesTikzLibrary{factory.safety.lidar}[2025/05/05 v1.0 Factory safety estlidarop device library]

\RequirePackage{tikz}
\usetikzlibrary{factory.safety}

\makeatletter

% \brief Add a new safety lidar device definition
% \param #1 : safety device ID
% \param #2 : safety device color
% \param #3 : safety device range
\newcommand{\addLidar}[3]{%
  \addSafety{lidar@#1}{0.5}{0.5}{#2}{#3}
}%

% \brief Draw a safety lidar device given a specific ID definition
% \param #1 : safety device ID
% \param #2 : safety device x-axis position
% \param #3 : safety device y-axis position
% \param #4 : safety device orientation [in degrees]
\newcommand{\drawLidar}[4]{%

  \drawSafety{lidar@#1}{#2}{#3}{#4}

  % Draw safety estop
  \begin{scope}[shift={(\item@x,\item@y)}, rotate=\item@orientation]

    \pgfmathsetmacro{\radius}{1*\item@length/3}%

    % Draw safety device
    \draw[fill=\item@color, draw=black] (\item@xorigin, \item@yorigin) rectangle (\item@xend, \item@yend);
    \draw[fill=black!60, draw=black] (0,0) circle (\radius);

    % Draw signal
    \draw[draw=red, ultra thick] (0,-1) arc (-90:90:1);
    \draw[draw=red, ultra thick] (0,-2) arc (-90:90:2);
    \draw[draw=red, ultra thick] (0,1) -- (0,2);
    \draw[draw=red, ultra thick] (0,-1) -- (0,-2);
    
  \end{scope}%
}

\makeatother