% tikzlibraryfactory.box.code.tex
\makeatletter
% If \ProvidesTikzLibrary isn't defined yet, define it as an alias for \ProvidesPackage
\@ifundefined{ProvidesTikzLibrary}{%
  \def\ProvidesTikzLibrary#1[#2]{%
    \ProvidesFile{tikzlibrary#1}[#2]%
  }%
}{}
\makeatother

\ProvidesTikzLibrary{factory.safety.fence}[2025/05/05 v1.0 Factory safety fence library]

\RequirePackage{tikz}
\usetikzlibrary{factory.item}
\usetikzlibrary{calc}

\makeatletter

% \brief Add a new safety fence definition
% \param #1 : safety device ID
% \param #2 : safety device length
% \param #3 : safety device width
% \param #4 : safety device color
\newcommand{\addFence}[4]{%
  \addItem{fence@#1}{#2}{#3}{#4}
}%

% \brief Draw a safety fence given a specific ID definition
% \param #1 : safety device ID
% \param #2 : safety device x-axis position
% \param #3 : safety device y-axis position
% \param #4 : safety device orientation [in degrees]
\newcommand{\drawFence}[4]{%

  % Get all parameters
  \drawItem{fence@#1}{#2}{#3}{#4}

  % Draw fence
  \begin{scope}[shift={(\item@x,\item@y)}, rotate=\item@orientation]

    \edef\fence@thickness{0.2}%

    % Draw fence
    \draw[fill=\item@color, draw=black, line width=1pt] (\item@xorigin, \item@yorigin) rectangle (\item@xend, \item@yend);%
    \draw[fill=white, draw=black, line width=1pt] (\item@xorigin+\fence@thickness, \item@yorigin + \fence@thickness) rectangle (\item@xend - \fence@thickness, (\item@yend - \fence@thickness);%
  \end{scope}%
}

\makeatother