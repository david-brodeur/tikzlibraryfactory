% tikzlibraryfactory.sensor.loadcell.code.tex
\makeatletter
% If \ProvidesTikzLibrary isn't defined yet, define it as an alias for \ProvidesPackage
\@ifundefined{ProvidesTikzLibrary}{%
  \def\ProvidesTikzLibrary#1[#2]{%
    \ProvidesFile{tikzlibrary#1}[#2]%
  }%
}{}
\makeatother

\ProvidesTikzLibrary{factory.sensor.loadcell}[2025/05/05 v1.0 Factory loadcell sensor library]

\RequirePackage{tikz}
\usetikzlibrary{factory.sensor}

\makeatletter

% \brief Add a new loadcell sensor definition
% \param #1 : sensor ID
% \param #2 : sensor color
% \param #3 : sensor range
\newcommand{\addLoadcell}[3]{%
  \addSensor{loadcell@#1}{0.4}{0.4}{#2}{#3}
}%

% \brief Draw a loadcell sensor given a specific ID definition
% \param #1 : sensor ID
% \param #2 : sensor x-axis position
% \param #3 : sensor y-axis position
% \param #4 : sensor orientation [in degrees]
\newcommand{\drawLoadcell}[4]{%

  \drawSensor{loadcell@#1}{#2}{#3}{#4}

  % Draw proximity sensor
  \begin{scope}[shift={(\item@x,\item@y)}, rotate=\item@orientation]

    % Draw sensor
    \draw[fill=black!60, draw=black, line width=0.3pt] 
      (\item@xorigin, \item@yorigin) rectangle (\item@xend, \item@yend);%
    \draw[draw=black, ultra thick] 
      (\signal@xorigin, \signal@yorigin) -- (\signal@xend, \signal@yend);%
    \draw[fill=black, draw=black, line width=0.3pt] 
      (\signal@xend, \signal@yend) circle (0.2*\item@width);%

    % Draw signal
    \draw[draw=\item@color!60, ultra thick] 
      (\signal@xend, \signal@yend) circle (0.5*\item@width);%
    \draw[draw=\item@color!60, ultra thick] 
      (\signal@xend, \signal@yend) circle (0.8*\item@width);%
    
  \end{scope}%
}

\makeatother