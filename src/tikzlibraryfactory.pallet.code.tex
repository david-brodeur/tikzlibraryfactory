% tikzlibraryfactory.pallet.code.tex
\makeatletter
% If \ProvidesTikzLibrary isn't defined yet, define it as an alias for \ProvidesPackage
\@ifundefined{ProvidesTikzLibrary}{%
  \def\ProvidesTikzLibrary#1[#2]{%
    \ProvidesFile{tikzlibrary#1}[#2]%
  }%
}{}
\makeatother

\ProvidesTikzLibrary{factory.pallet}[2025/05/05 v1.0 Factory pallet library]

\RequirePackage{tikz}
\usetikzlibrary{factory.item}
\usetikzlibrary{calc}

\makeatletter

% \brief Add a new pallet definition
% \param #1 : pallet ID
% \param #2 : pallet length
% \param #3 : pallet width
% \param #4 : pallet color
\newcommand{\addPallet}[4]{%
  \addItem{pallet@#1}{#2}{#3}{#4}
}%

% \brief Draw a pallet given a specific ID definition
% \param #1 : pallet ID
% \param #2 : pallet x-axis position
% \param #3 : pallet y-axis position
% \param #4 : pallet orientation [in degrees]
\newcommand{\drawPallet}[4]{%

  % Get all parameters
  \drawItem{pallet@#1}{#2}{#3}{#4}
  
  % Compute wooden boards coordinates
  \pgfmathsetmacro{\step}{0.5}%
  \pgfmathsetmacro{\woodenBoardWidth}{3 * \step / 5}%
  \pgfmathsetmacro{\halfWoodenBoardWidth}{\woodenBoardWidth / 2}%

  % Draw pallet
  \begin{scope}[shift={(\item@x,\item@y)}, rotate=\item@orientation]

    \draw[draw=black, line width=2pt]
        ($(\item@xorigin, \item@yorigin) + (- \halfWoodenBoardWidth, 0)$) rectangle ($(\item@xend, \item@yend) + (\halfWoodenBoardWidth, 0)$);%

    % Draw each wooden board
    \pgfmathsetmacro{\N}{\item@width / \step - 1}%
    \foreach \i in {1,...,\N} {
      \draw[fill=\item@color!60, draw=black, line width=0.3pt]
        ($(\item@xorigin, \item@yorigin) + (0, \i*\step - \woodenBoardWidth/2)$) rectangle ($(\item@xend, \item@yorigin) + (0, \i*\step + \woodenBoardWidth/2)$);%
    }

    \pgfmathsetmacro{\N}{\item@length / \step}%
    \foreach \i in {0,...,\N} {
      \draw[fill=\item@color!60, draw=black, line width=0.3pt]
        ($(\item@xorigin, \item@yorigin) + (\i*\step - \halfWoodenBoardWidth, 0)$) rectangle ($(\item@xorigin, \item@yend) + (\i*\step + \halfWoodenBoardWidth, 0)$);%
    }
      
  \end{scope}%
}

\makeatother