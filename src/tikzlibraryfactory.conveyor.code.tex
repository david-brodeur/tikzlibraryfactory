% tikzlibraryfactory.conveyor.code.tex
\makeatletter
% If \ProvidesTikzLibrary isn't defined yet, define it as an alias for \ProvidesPackage
\@ifundefined{ProvidesTikzLibrary}{%
  \def\ProvidesTikzLibrary#1[#2]{%
    \ProvidesFile{tikzlibrary#1}[#2]%
  }%
}{}
\makeatother

\ProvidesTikzLibrary{factory.conveyor}[2025/05/05 v1.0 Factory conveyor library]

\RequirePackage{tikz}
\usetikzlibrary{factory.item}
\usetikzlibrary{calc}

\makeatletter

% \brief Add a new conveyor definition
% \param #1 : conveyor ID
% \param #2 : conveyor length
% \param #3 : conveyor width
% \param #4 : conveyor color
\newcommand{\addConveyor}[4]{%
  \addItem{conveyor@#1}{#2}{#3}{#4}
}%

% \brief Draw a conveyor given a specific ID definition
% \param #1 : conveyor ID
% \param #2 : conveyor x-axis position
% \param #3 : conveyor y-axis position
% \param #4 : conveyor orientation [in degrees]
\newcommand{\drawConveyor}[4]{%

  % Get all parameters
  \drawItem{conveyor@#1}{#2}{#3}{#4}
  
  % Compute conveyor frame coordinates
  \pgfmathsetmacro{\item@xorigin}{0}%
  \pgfmathsetmacro{\item@yorigin}{0 - (\item@width/2)}%
  \pgfmathsetmacro{\item@xend}{\item@length}%
  \pgfmathsetmacro{\item@yend}{(\item@width/2)}%

  % Compute conveyor rollers dimensions
  \pgfmathsetmacro{\cornerRadius}{1.5}%
  \pgfmathsetmacro{\step}{0.25}%
  \pgfmathsetmacro{\N}{\item@length / \step - 1}%
  \pgfmathsetmacro{\rollerDiameter}{3 * \step / 5}%
  \pgfmathsetmacro{\rollerRadius}{\rollerDiameter / 2}%

  % Draw conveyor
  \begin{scope}[shift={(\item@x,\item@y)}, rotate=\item@orientation]

    % Draw frame
    \draw[fill=white, draw=black, line width=0.3pt] 
      (\item@xorigin, \item@yorigin) rectangle (\item@xend, (\item@yend);%

    % Draw rollers
    \foreach \i in {1,...,\N} {
      \draw[rounded corners=\cornerRadius, fill=\item@color!60, draw=black!70, line width=0.3pt]
        ($(\item@xorigin, \item@yorigin) + (\i*\step - \rollerRadius, 0)$) rectangle ($(\item@xorigin, \item@yend) + (\i*\step + \rollerRadius, 0)$);%
    }
    
  \end{scope}%
}

\makeatother