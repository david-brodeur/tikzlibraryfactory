% tikzlibraryfactory.sensor.proximity.code.tex
\makeatletter
% If \ProvidesTikzLibrary isn't defined yet, define it as an alias for \ProvidesPackage
\@ifundefined{ProvidesTikzLibrary}{%
  \def\ProvidesTikzLibrary#1[#2]{%
    \ProvidesFile{tikzlibrary#1}[#2]%
  }%
}{}
\makeatother

\ProvidesTikzLibrary{factory.sensor.proximity}[2025/05/05 v1.0 Factory proximity sensor library]

\RequirePackage{tikz}
\usetikzlibrary{factory.sensor}

\makeatletter

% \brief Add a new proximity sensor definition
% \param #1 : sensor ID
% \param #2 : sensor color
% \param #3 : sensor range
\newcommand{\addProximitySensor}[3]{%
  \addSensor{proximity@#1}{0.4}{0.25}{#2}{#3}
}%

% \brief Draw a proximity sensor given a specific ID definition
% \param #1 : sensor ID
% \param #2 : sensor x-axis position
% \param #3 : sensor y-axis position
% \param #4 : sensor orientation [in degrees]
\newcommand{\drawProximitySensor}[4]{%

  \drawSensor{proximity@#1}{#2}{#3}{#4}

  % Draw proximity sensor
  \begin{scope}[shift={(\item@x,\item@y)}, rotate=\item@orientation]

    % Draw sensor
    \draw[fill=black!60, draw=black, line width=0.3pt] 
      (\item@xorigin, \item@yorigin) rectangle (\item@xend, \item@yend);%

    % Draw signal
    \draw[draw=\item@color!60, dashed, ultra thick] 
      (\signal@xorigin, \signal@yorigin) -- (\signal@xend, \signal@yend);%
    
  \end{scope}%
}

\makeatother