% tikzlibraryfactory.sensor.camera.code.tex
\makeatletter
% If \ProvidesTikzLibrary isn't defined yet, define it as an alias for \ProvidesPackage
\@ifundefined{ProvidesTikzLibrary}{%
  \def\ProvidesTikzLibrary#1[#2]{%
    \ProvidesFile{tikzlibrary#1}[#2]%
  }%
}{}
\makeatother

\ProvidesTikzLibrary{factory.sensor.camera}[2025/05/05 v1.0 Factory camera sensor library]

\RequirePackage{tikz}
\usetikzlibrary{factory.sensor}
\usetikzlibrary{shapes}

\makeatletter

% \brief Add a new camera sensor definition
% \param #1 : sensor ID
% \param #2 : sensor color
% \param #3 : sensor range
\newcommand{\addCamera}[3]{%
  \addSensor{camera@#1}{0.4}{0.4}{#2}{#3}
}%

% \brief Draw a camera sensor given a specific ID definition
% \param #1 : sensor ID
% \param #2 : sensor x-axis position
% \param #3 : sensor y-axis position
% \param #4 : sensor orientation [in degrees]
\newcommand{\drawCamera}[4]{%

  \drawSensor{camera@#1}{#2}{#3}{#4}

  % Draw camera sensor
  \begin{scope}[shift={(\item@x,\item@y)}, rotate=\item@orientation]

    % Draw lens
    \edef\camera@lensThickness{0.2}
    \edef\camera@lensWidth{0.8*\item@width}
    \pgfmathsetmacro{\camera@lensOrigin}{\item@xend + \camera@lensThickness}
    
    \drawTriangle{\camera@lensWidth}{\camera@lensOrigin}{90}{black!60}

    % Draw sensor
    \draw[fill=black!60, draw=black, line width=0.3pt] 
      (\item@xorigin, \item@yorigin) rectangle (\item@xend, \item@yend);%

    % Draw signal
    \draw[draw=\item@color, ultra thick] 
      (\camera@lensOrigin, \signal@yorigin-\camera@lensWidth/2) 
      -- (\signal@xend, \signal@yorigin-2*\camera@lensWidth) 
      -- (\signal@xend, \signal@yorigin+2*\camera@lensWidth) 
      -- (\camera@lensOrigin, \signal@yorigin+\camera@lensWidth/2);%

    
  \end{scope}%
}

\makeatother