% tikzlibraryfactory.safety.estop.code.tex
\makeatletter
% If \ProvidesTikzLibrary isn't defined yet, define it as an alias for \ProvidesPackage
\@ifundefined{ProvidesTikzLibrary}{%
  \def\ProvidesTikzLibrary#1[#2]{%
    \ProvidesFile{tikzlibrary#1}[#2]%
  }%
}{}
\makeatother

\ProvidesTikzLibrary{factory.safety.estop}[2025/05/05 v1.0 Factory safety estop device library]

\RequirePackage{tikz}
\usetikzlibrary{factory.safety}

\makeatletter

% \brief Add a new safety estop device definition
% \param #1 : safety device ID
% \param #2 : safety device color
% \param #3 : safety device range
\newcommand{\addEStop}[3]{%
  \addSafety{estop@#1}{0.3}{0.3}{#2}{#3}
}%

% \brief Draw a safety estop device given a specific ID definition
% \param #1 : safety device ID
% \param #2 : safety device x-axis position
% \param #3 : safety device y-axis position
% \param #4 : safety device orientation [in degrees]
\newcommand{\drawEStop}[4]{%

  \drawSafety{estop@#1}{#2}{#3}{#4}

  % Draw safety estop
  \begin{scope}[shift={(\item@x,\item@y)}, rotate=\item@orientation]

    \pgfmathsetmacro{\radius}{1*\item@length/3}%
    \pgfmathsetmacro{\lightRadius}{2*\radius/3}%

    % Draw safety device
    \draw[fill=\item@color, draw=black] (\item@xorigin, \item@yorigin) rectangle (\item@xend, \item@yend);
    \draw[fill=red!60, draw=black] (0,0) circle (\radius);
    \draw[fill=red, draw=black] (0,0) circle (\lightRadius);
    
  \end{scope}%
}

\makeatother