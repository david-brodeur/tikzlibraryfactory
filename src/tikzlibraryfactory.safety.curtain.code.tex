% tikzlibraryfactory.safety.curtain.code.tex
\makeatletter
% If \ProvidesTikzLibrary isn't defined yet, define it as an alias for \ProvidesPackage
\@ifundefined{ProvidesTikzLibrary}{%
  \def\ProvidesTikzLibrary#1[#2]{%
    \ProvidesFile{tikzlibrary#1}[#2]%
  }%
}{}
\makeatother

\ProvidesTikzLibrary{factory.safety.curtain}[2025/05/05 v1.0 Factory safety curtain device library]

\RequirePackage{tikz}
\usetikzlibrary{factory.safety}

\makeatletter

% \brief Add a new safety curtain device definition
% \param #1 : safety device ID
% \param #2 : safety device color
% \param #3 : safety device range
\newcommand{\addCurtain}[3]{%
  \addSafety{curtain@#1}{0.3}{0.3}{#2}{#3}
}%

% \brief Draw a safety curtain device given a specific ID definition
% \param #1 : safety device ID
% \param #2 : safety device x-axis position
% \param #3 : safety device y-axis position
% \param #4 : safety device orientation [in degrees]
\newcommand{\drawCurtain}[4]{%

  \drawSafety{curtain@#1}{#2}{#3}{#4}

  % Draw safety curtain
  \begin{scope}[shift={(\item@x,\item@y)}, rotate=\item@orientation]

    \pgfmathsetmacro{\baseRadius}{\item@length/2}%
    \pgfmathsetmacro{\curtainRadius}{2*\baseRadius/3}%

    % Draw safety device
    \draw[fill=black] (0,0) circle (\baseRadius);
    \draw[fill=\item@color, draw=black] (0,0) circle (\curtainRadius);

    \draw[fill=black] (\safety@range,0) circle (\baseRadius);
    \draw[fill=\item@color, draw=black] (\safety@range,0) circle (\curtainRadius);

    % Draw signal
    \draw[draw=red, dashed, ultra thick] (\curtainRadius, 0) -- (\safety@range - \curtainRadius, 0);
    
  \end{scope}%
}

\makeatother