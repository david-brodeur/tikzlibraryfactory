% tikzlibraryfactory.pallet.code.tex
\makeatletter
% If \ProvidesTikzLibrary isn't defined yet, define it as an alias for \ProvidesPackage
\@ifundefined{ProvidesTikzLibrary}{%
  \def\ProvidesTikzLibrary#1[#2]{%
    \ProvidesFile{tikzlibrary#1}[#2]%
  }%
}{}
\makeatother

\ProvidesTikzLibrary{factory.robot.agv}[2025/05/05 v1.0 Factory agv robot library]

\RequirePackage{tikz}
\usetikzlibrary{calc}
\usetikzlibrary{factory.robot}

\makeatletter

% \brief Add a new agv robot definition
% \param #1 : robot ID
% \param #2 : robot length
% \param #3 : robot width
% \param #4 : robot color
% \param #5 : robot front wheels orientation
\newcommand{\addAGV}[5]{%
  \addRobot{agv@#1}{#2}{#3}{#4}
  \expandafter\gdef\csname robot@agv@#1@wheelsOrientation\endcsname{#5}%
}%

% \brief Draw a robot given a specific ID definition
% \param #1 : robot ID
% \param #2 : robot x-axis position
% \param #3 : robot y-axis position
% \param #4 : robot orientation [in degrees]
\newcommand{\drawAGV}[4]{%

  \drawRobot{agv@#1}{#2}{#3}{#4}
  \edef\robot@wheelsOrientation{\csname robot@agv@#1@wheelsOrientation\endcsname}%

  % Draw robot
  \begin{scope}[shift={(\item@x,\item@y)}, rotate=\item@orientation]

    \edef\camera@cornerRadius{10}%
    \edef\camera@wheelOffset{0.3}%
    \edef\camera@wheelRadius{0.3}%
    \edef\camera@wheelThickness{0.4}%

    \pgfmathsetmacro{\wheel@front@x}{\item@xend - \camera@wheelOffset - \camera@wheelRadius}%
    \pgfmathsetmacro{\wheel@front@y}{\item@yend}%

    \pgfmathsetmacro{\wheel@rear@x}{\item@xorigin + \camera@wheelOffset + \camera@wheelRadius}%
    \pgfmathsetmacro{\wheel@rear@y}{\item@yend}%

    %Draw wheels
    \foreach \i in {-1,1} {

      % Draw front wheels
      \begin{scope}[shift={(\wheel@front@x, \i*\wheel@front@y)}, rotate=\robot@wheelsOrientation]
        \draw[fill=black!60, draw=black] (-\camera@wheelRadius, -\camera@wheelThickness/2) rectangle (\camera@wheelRadius,\camera@wheelThickness/2);
      \end{scope}

      % Draw rear wheels
      \begin{scope}[shift={(\wheel@rear@x, \i*\wheel@rear@y)}, rotate=0]
        \draw[fill=black!60, draw=black] (-\camera@wheelRadius, -\camera@wheelThickness/2) rectangle (\camera@wheelRadius,\camera@wheelThickness/2);
      \end{scope}
    }

    % Draw robot base
    \draw[rounded corners=\camera@cornerRadius, fill=\item@color!60, draw=black, line width=0.3pt] (\item@xorigin, \item@yorigin) rectangle (\item@xend, (\item@yend);%
      
  \end{scope}%
}

\makeatother