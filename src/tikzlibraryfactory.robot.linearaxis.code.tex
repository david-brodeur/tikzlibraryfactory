% tikzlibraryfactory.robot.linearaxis.code.tex
\makeatletter
% If \ProvidesTikzLibrary isn't defined yet, define it as an alias for \ProvidesPackage
\@ifundefined{ProvidesTikzLibrary}{%
  \def\ProvidesTikzLibrary#1[#2]{%
    \ProvidesFile{tikzlibrary#1}[#2]%
  }%
}{}
\makeatother

\ProvidesTikzLibrary{factory.box}[2025/05/05 v1.0 Factory linearaxis library]

\RequirePackage{tikz}
\RequirePackage{ifthen}
\usetikzlibrary{factory.robot}
\usetikzlibrary{calc}

\makeatletter

% \brief Add a new linear axis definition
% \param #1 : robot ID
% \param #2 : robot length
% \param #3 : robot width
% \param #4 : robot color
\newcommand{\addLinearAxis}[4]{%
  \addRobot{box@#1}{#2}{#3}{#4}
}%

% \brief Draw a linear axis given a specific ID definition
% \param #1 : robot ID
% \param #2 : robot x-axis position
% \param #3 : robot y-axis position
% \param #4 : robot orientation [in degrees]
\newcommand{\drawLinearAxis}[4]{%

  % Get all parameters
  \drawRobot{box@#1}{#2}{#3}{#4}

  % Draw item
  \begin{scope}[shift={(\item@x,\item@y)}, rotate=\item@orientation]

    % Draw linear axis
    \draw[fill=\item@color, draw=black, line width=0.3pt] (\item@xorigin, \item@yorigin) rectangle (\item@xend, (\item@yend);%
    \draw[black, dashed, thick] (\item@xorigin, 0) -- (\item@xend, 0);

  \end{scope}%
}

\makeatother