% tikzlibraryfactory.pallet.code.tex
\makeatletter
% If \ProvidesTikzLibrary isn't defined yet, define it as an alias for \ProvidesPackage
\@ifundefined{ProvidesTikzLibrary}{%
  \def\ProvidesTikzLibrary#1[#2]{%
    \ProvidesFile{tikzlibrary#1}[#2]%
  }%
}{}
\makeatother

\ProvidesTikzLibrary{factory.robot.arm}[2025/05/05 v1.0 Factory robot arm library]

\RequirePackage{tikz}
\RequirePackage{xfp}
\RequirePackage{ifthen}
\RequirePackage{pgf}
\usetikzlibrary{calc}
\usetikzlibrary{shadings}

\makeatletter

% ==========
% PARAMÈTRES GLOBAUX
% ==========

\pgfkeys{
  /robot/.is family,
  /robot,
  id/.store in=\id, id/.initial=robot,
  x/.store in=\x, x/.initial=0.0,
  y/.store in=\y, y/.initial=0.0,
  nJoints/.store in=\nJoints, nJoints/.initial=2,
  linkColor/.store in=\linkColor, linkColor/.initial=white,
  isShading/.store in=\isShading, isShading/.initial=0,
  isTool/.store in=\isTool, isTool/.initial=1,
  isWorkspace/.store in=\isWorkspace, isWorkspace/.initial=0,
  workspaceRadius/.store in=\workspaceRadius, workspaceRadius/.initial=0
}

\newcommand{\linkThickness}{0.4}       % Épaisseur des bras
\newcommand{\linkCornerRadius}{3.5pt}  % Rayon des coins arrondis
\newcommand{\robotBaseRadius}{0.35}    % Rayon de la base

% ==========
% CONFIGURATION ROBOT
% ==========

% Enregistrer les paramètres DH : theta et a
\newcommand{\setArmDHParameters}[5]{%
  \expandafter\def\csname theta@#1@#2\endcsname{#3}%
  \expandafter\def\csname a@#1@#2\endcsname{#4}%
}

% Initialiser un robot à une position donnée
\newcommand{\initRobotArm}[3]{%
  \expandafter\def\csname xbase@#1\endcsname{#2}%
  \expandafter\def\csname ybase@#1\endcsname{#3}%
  \expandafter\def\csname xprev@#1\endcsname{#2}%
  \expandafter\def\csname yprev@#1\endcsname{#3}%
  \expandafter\def\csname angleprev@#1\endcsname{0}%
}

% Calculer la position du lien courant
\newcommand{\computeArmLinkPosition}[3]{%
  \pgfmathsetmacro{\theta}{\csname theta@#1@#2\endcsname}%
  \pgfmathsetmacro{\a}{\csname a@#1@#2\endcsname}%
  \pgfmathsetmacro{\angleprev}{\csname angleprev@#1\endcsname}%
  \pgfmathsetmacro{\xprev}{\csname xprev@#1\endcsname}%
  \pgfmathsetmacro{\yprev}{\csname yprev@#1\endcsname}%
  \pgfmathsetmacro{\angletot}{\angleprev + \theta}%
  \pgfmathsetmacro{\xnew}{\xprev + (\a - #3) * cos(\angletot)}%
  \pgfmathsetmacro{\ynew}{\yprev + (\a - #3) * sin(\angletot)}%
}

% Mettre à jour les coordonnées après chaque lien
\newcommand{\updateRobotArmState}[1]{%
  \expandafter\xdef\csname xprev@#1\endcsname{\xnew}%
  \expandafter\xdef\csname yprev@#1\endcsname{\ynew}%
  \expandafter\xdef\csname angleprev@#1\endcsname{\angletot}%
}

% Dessiner un lien
\newcommand{\drawArmLink}[5]{%
  \computeArmLinkPosition{#1}{#2}{0.25}%
  \pgfmathsetmacro{\theta}{\csname theta@#1@#2\endcsname}%
  \pgfmathsetmacro{\a}{\csname a@#1@#2\endcsname}%
  \pgfmathsetmacro{\midx}{(\xprev+\xnew)/2}%
  \pgfmathsetmacro{\midy}{(\yprev+\ynew)/2}%

  \begin{scope}
    \pgftransformshift{\pgfpoint{\midx cm}{\midy cm}}%
    \pgftransformrotate{\angleprev + \theta}%
    \ifthenelse{\equal{#4}{true}}{%
      \shade[rounded corners=\linkCornerRadius, left color=#3!80, right color=#3!20, draw=black, line width=0.3pt] (-\a/2, -#5/2) rectangle (\a/2, #5/2);%
    }{%
      \draw[rounded corners=\linkCornerRadius, fill=#3!60, draw=black, line width=0.3pt] (-\a/2, -#5/2) rectangle (\a/2, #5/2);%
    }%
  \end{scope}

  \updateRobotArmState{#1}%
}

% Dessiner la base
\newcommand{\drawArmBase}[3]{%
  \draw[fill=gray!50, draw=black] (#2,#3) circle (#1);
}

% Dessiner une pince
\newcommand{\drawArmGripper}[1]{%
  \pgfmathsetmacro{\gripperSize}{0.1}%
  \pgfmathsetmacro{\gripperLength}{0.5}%
  \pgfmathsetmacro{\a}{0.4}%

  \pgfmathsetmacro{\angleprev}{\csname angleprev@#1\endcsname}%
  \pgfmathsetmacro{\xprev}{\csname xprev@#1\endcsname}%
  \pgfmathsetmacro{\yprev}{\csname yprev@#1\endcsname}%

  \pgfmathsetmacro{\xnew}{\xprev + (\a - \gripperSize) * cos(\angleprev)}%
  \pgfmathsetmacro{\ynew}{\yprev + (\a - \gripperSize) * sin(\angleprev)}%

  \pgfmathsetmacro{\midx}{(\xprev+\xnew)/2}%
  \pgfmathsetmacro{\midy}{(\yprev+\ynew)/2}%

  \begin{scope}
    \pgftransformshift{\pgfpoint{\midx cm}{\midy cm}}%
    \pgftransformrotate{\angleprev}%
    \draw[fill=gray!50, draw=black, line width=0.3pt] (-\gripperSize/2, -\gripperLength/2) rectangle (\gripperSize/2, \gripperLength/2);%
    \draw[fill=gray!50, draw=black, line width=0.3pt] (\gripperSize/2, -\gripperLength/2+0.05) rectangle (\gripperSize/2+0.3, -\gripperLength/2+0.15);%
    \draw[fill=gray!50, draw=black, line width=0.3pt] (\gripperSize/2, \gripperLength/2-0.05) rectangle (\gripperSize/2+0.3, \gripperLength/2-0.15);%
  \end{scope}%

  \expandafter\xdef\csname xprev@#1\endcsname{\xnew}%
  \expandafter\xdef\csname yprev@#1\endcsname{\ynew}%
  \expandafter\xdef\csname angleprev@#1\endcsname{\angleprev}%
}

% Dessin complet d’un robot
\newcommand{\drawRobotArm}[1][]{%
  % Récupération des paramètres via les clés /robot:
  %   \id, \x, \y, \nJoints, \linkColor, \isShading, \isTool,
  %   \isWorkspace, \workspaceRadius
  \pgfkeys{/robot, #1}%
  \initRobotArm{\id}{\x}{\y}%
  \drawArmBase{\robotBaseRadius}{\x}{\y}%
  
  % Initialisation d'une variable globale pour calculer le rayon total
  \xdef\radiusAuto{0}%
  
  \foreach \i in {1,...,\nJoints} {%
    \pgfmathsetmacro{\a}{\csname a@\id @\i\endcsname}%
    \pgfmathparse{\radiusAuto + \a}%
    \xdef\radiusAuto{\pgfmathresult}%
    \drawArmLink{\id}{\i}{\linkColor}{\isShading}{\linkThickness}%
  }%
  
  % Dessiner la pince si activée
  \ifthenelse{\equal{\isTool}{true}}{%
    \drawArmGripper{\id}%
  }{}%
  
  % Dessiner le workspace s'il est activé, en sélectionnant le rayon approprié
  \ifthenelse{\equal{\isWorkspace}{true}}{%
    \pgfmathparse{\workspaceRadius == 0 ? \radiusAuto : \workspaceRadius}%
    \let\radiusToUse\pgfmathresult%
    \draw[gray, dashed] (\x,\y) circle (\radiusToUse);%
  }{}%
}

\makeatother