% tikzlibraryfactory.robot.arm.code.tex
\makeatletter
% If \ProvidesTikzLibrary isn't defined yet, define it as an alias for \ProvidesPackage
\@ifundefined{ProvidesTikzLibrary}{%
  \def\ProvidesTikzLibrary#1[#2]{%
    \ProvidesFile{tikzlibrary#1}[#2]%
  }%
}{}
\makeatother

\ProvidesTikzLibrary{factory.robot.arm}[2025/05/05 v1.0 Factory robot arm library]

\RequirePackage{tikz}
\RequirePackage{xfp}
\RequirePackage{ifthen}
\RequirePackage{pgf}
\usetikzlibrary{factory.robot}
\usetikzlibrary{calc}
\usetikzlibrary{shadings}

\makeatletter

% \brief Set Denavit Hartenberg parameters for a specified link
% \param #1 : robot ID
% \param #2 : link ID
% \param #3 : Denavit Hartenberg a parameter
% \param #4 : Denavit Hartenberg d parameter
% \param #5 : Denavit Hartenberg theta parameter
\newcommand{\setDHParameters}[5]{%
  \expandafter\gdef\csname robot@arm@#1@#2@a\endcsname{#3}%
  \expandafter\gdef\csname robot@arm@#1@#2@d\endcsname{#4}%
  \expandafter\gdef\csname robot@arm@#1@#2@theta\endcsname{#5}%
}

% \brief Add a new robot link definition
% \param #1 : robot ID
% \param #2 : link ID
% \param #3 : Denavit Hartenberg a parameter
% \param #4 : Denavit Hartenberg d parameter
% \param #5 : Denavit Hartenberg theta parameter
\newcommand{\addLink}[5]{%
  \setDHParameters{#1}{#2}{#3}{#4}{#5}
}

% \brief Add a new robotic arm definition
% \param #1 : robot ID
% \param #2 : number of links
% \param #3 : color
% \param #4 : is color shading enabled
% \param #5 : is tool visible
% \param #6 : is workspace visible

\newcommand{\addArm}[6]{%
  \addRobot{arm@#1}{0}{0}{#3}
  \expandafter\gdef\csname robot@arm@#1@nLinks\endcsname{#2}%
  \expandafter\gdef\csname robot@arm@#1@isShading\endcsname{#4}%
  \expandafter\gdef\csname robot@arm@#1@isTool\endcsname{#5}%
  \expandafter\gdef\csname robot@arm@#1@isWorkspace\endcsname{#6}%
}

% \brief Initialize pose
% \param #1 : robot ID
% \param #2 : robot's first link x-axis position
% \param #3 : robot's first link y-axis position
% \param #4 : robot's first link orientation
\newcommand{\initializePose}[4]{%
  \expandafter\def\csname robot@#1@xbase\endcsname{#2}%
  \expandafter\def\csname robot@#1@ybase\endcsname{#3}%
  \expandafter\def\csname robot@#1@xprev\endcsname{#2}%
  \expandafter\def\csname robot@#1@yprev\endcsname{#3}%
  \expandafter\def\csname robot@#1@angleprev\endcsname{#4}%
}

% \brief Compute current link's pose
% \param #1 : robot ID
% \param #2 : link ID
% \param #3 : link overlap
\newcommand{\computeLinkPose}[3]{%

  \edef\link@a{\csname robot@#1@#2@a\endcsname}%
  \edef\link@theta{\csname robot@#1@#2@theta\endcsname}%

  \edef\link@xprev{\csname robot@#1@xprev\endcsname}%
  \edef\link@yprev{\csname robot@#1@yprev\endcsname}%
  \edef\link@angleprev{\csname robot@#1@angleprev\endcsname}%
  
  \pgfmathsetmacro{\angletot}{\link@angleprev + \link@theta}%
  \pgfmathsetmacro{\xnew}{\link@xprev + (\link@a - #3) * cos(\angletot)}%
  \pgfmathsetmacro{\ynew}{\link@yprev + (\link@a - #3) * sin(\angletot)}%
}

% \brief Update pose
% \param #1 : robot ID
% \param #2 : new xprev value
% \param #3 : new yprev value
% \param #4 : new angleprev value
\newcommand{\updatePose}[4]{%
  \expandafter\xdef\csname robot@#1@xprev\endcsname{#2}%
  \expandafter\xdef\csname robot@#1@yprev\endcsname{#3}%
  \expandafter\xdef\csname robot@#1@angleprev\endcsname{#4}%
}

% \brief Draw a robot link
% \param #1 : robot ID
% \param #2 : robot link ID
% \param #3 : robot link color
% \param #4 : is robot link shading
% \param #5 : robot link thickness
\newcommand{\drawLink}[5]{%

  \edef\link@overlap{0.25}%
  \edef\link@cornerRadius{3.5}%
  \edef\link@xprev{\csname robot@#1@xprev\endcsname}%
  \edef\link@yprev{\csname robot@#1@yprev\endcsname}%
  \computeLinkPose{#1}{#2}{\link@overlap}%
  
  \pgfmathsetmacro{\link@xorigin}{(\link@xprev+\xnew)/2}%
  \pgfmathsetmacro{\link@yorigin}{(\link@yprev+\ynew)/2}%

  \begin{scope}[shift={(\link@xorigin,\link@yorigin)}, rotate={\angletot}]

    \ifthenelse{\equal{#4}{true}}{%
      \shade[rounded corners=\link@cornerRadius, left color=#3!80, right color=#3!20, draw=black, line width=0.3pt] (-\link@a/2, -#5/2) rectangle (\link@a/2, #5/2);%
    }{%
      \draw[rounded corners=\link@cornerRadius, fill=#3!60, draw=black, line width=0.3pt] (-\link@a/2, -#5/2) rectangle (\link@a/2, #5/2);%
    }%
  \end{scope}
%
  \updatePose{#1}{\xnew}{\ynew}{\angletot}%
}

% \brief Draw a robot base
% \param #1 : robot base x-axis position
% \param #2 : robot base y-axis position
% \param #3 : robot base radius
\newcommand{\drawBase}[3]{%
  \draw[fill=gray!50, draw=black] (#1,#2) circle (#3);
}

% Dessiner une pince
\newcommand{\drawTool}[1]{%

  \edef\tool@a{0.4}%
  \edef\tool@theta{0}%
  \edef\tool@size{0.1}%
  \edef\tool@length{0.5}%

  \edef\tool@angleprev{\csname robot@#1@angleprev\endcsname}%
  \edef\tool@xprev{\csname robot@#1@xprev\endcsname}%
  \edef\tool@yprev{\csname robot@#1@yprev\endcsname}%

  \pgfmathsetmacro{\angletot}{\tool@angleprev + \tool@theta}%
  \pgfmathsetmacro{\xnew}{\tool@xprev + (\tool@a - \tool@size) * cos(\angletot)}%
  \pgfmathsetmacro{\ynew}{\tool@yprev + (\tool@a - \tool@size) * sin(\angletot)}%

  \pgfmathsetmacro{\tool@xorigin}{(\tool@xprev+\xnew)/2}%
  \pgfmathsetmacro{\tool@yorigin}{(\tool@yprev+\ynew)/2}%

  \begin{scope}[shift={(\tool@xorigin,\tool@yorigin)}, rotate={\angletot}]
    \draw[fill=gray!50, draw=black, line width=0.3pt] (-\tool@size/2, -\tool@length/2) rectangle (\tool@size/2, \tool@length/2);%
    \draw[fill=gray!50, draw=black, line width=0.3pt] (\tool@size/2, -\tool@length/2+0.05) rectangle (\tool@size/2+0.3, -\tool@length/2+0.15);%
    \draw[fill=gray!50, draw=black, line width=0.3pt] (\tool@size/2, \tool@length/2-0.05) rectangle (\tool@size/2+0.3, \tool@length/2-0.15);%
  \end{scope}%

  \updatePose{#1}{\xnew}{\ynew}{\angletot}%

}

% \brief Draw a robotic arm given a specific ID definition
% \param #1 : robot ID
% \param #2 : robot x-axis position
% \param #3 : robot y-axis position
% \param #4 : robot orientation [in degrees]
\newcommand{\drawArm}[4]{%

  \drawRobot{arm@#1}{#2}{#3}{#4}
  \edef\arm@nLinks{\csname robot@arm@#1@nLinks\endcsname}%
  \edef\arm@isShading{\csname robot@arm@#1@isShading\endcsname}
  \edef\arm@isTool{\csname robot@arm@#1@isTool\endcsname}
  \edef\arm@isWorkspace{\csname robot@arm@#1@isWorkspace\endcsname}

  \edef\arm@linkThickness{0.4}%
  
  % Draw robotic arm
  \begin{scope}[shift={(\item@x,\item@y)}, rotate=\item@orientation]

    \edef\arm@baseRadius{0.35}%
    \drawBase{0}{0}{\arm@baseRadius}


    \initializePose{arm@#1}{0}{0}{0}%
    \foreach \linkID in {1,...,\arm@nLinks} {%
      \drawLink{arm@#1}{\linkID}{\item@color}{\arm@isShading}{\arm@linkThickness}%
    }%

    \ifthenelse{\equal{\arm@isTool}{true}}{%
          \drawTool{arm@#1}
    }
    {}

    
      
  \end{scope}%
}

\makeatother