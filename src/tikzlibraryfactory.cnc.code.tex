% tikzlibraryfactory.cnc.code.tex
\makeatletter
% If \ProvidesTikzLibrary isn't defined yet, define it as an alias for \ProvidesPackage
\@ifundefined{ProvidesTikzLibrary}{%
  \def\ProvidesTikzLibrary#1[#2]{%
    \ProvidesFile{tikzlibrary#1}[#2]%
  }%
}{}
\makeatother

\ProvidesTikzLibrary{factory.cnc}[2025/05/05 v1.0 Factory cnc machines library]

\RequirePackage{tikz}
\usetikzlibrary{factory.item}
\usetikzlibrary{calc}

\makeatletter

% \brief Add a new cnc machine definition
% \param #1 : cnc ID
% \param #2 : cnc length
% \param #3 : cnc width
% \param #4 : cnc color
\newcommand{\addCNC}[4]{%
    \addItem{cnc@#1}{#2}{#3}{#4}
}%

% \brief Draw a cnc given a specific ID definition
% \param #1 : cnc ID
% \param #2 : cnc x-axis position
% \param #3 : cnc y-axis position
% \param #4 : cnc orientation [in degrees]
\newcommand{\drawCNC}[4]{%

  % Get all parameters
  \drawItem{cnc@#1}{#2}{#3}{#4}

  % Compute door coordinates
  \pgfmathsetmacro{\doorOffset}{(0.05}%
  \pgfmathsetmacro{\door@xorigin}{(0.05*\item@length) - \item@xoffset}%
  \pgfmathsetmacro{\door@yorigin}{0 - \item@yoffset - \doorOffset}%
  \pgfmathsetmacro{\door@xend}{(0.55*\item@length) - \item@xoffset}%
  \pgfmathsetmacro{\door@yend}{0.1*\item@width -\item@yoffset}%

  % Draw cnc
  \begin{scope}[shift={(\item@x,\item@y)}, rotate=\item@orientation]

    % Draw frame
    \draw[fill=\item@color!60, draw=black, line width=1pt] (\item@xorigin, \item@yorigin) rectangle (\item@xend, (\item@yend);%

    % Draw cnc's door
    \draw[fill=white, opacity=0.9, draw=black, line width=0.3pt] (\door@xorigin, \door@yorigin) rectangle (\door@xend, (\door@yend);%
    
  \end{scope}%
}

\makeatother